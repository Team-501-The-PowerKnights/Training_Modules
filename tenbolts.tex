\documentclass[letterpaper]{article}
\usepackage{outlines}

%\usepackage{fontspec}
%\defaultfontfeatures{Mapping=tex-text,Scale=MatchLowercase}
\setlength{\parindent}{0in}

\begin{document}
\fontfamily{lmtt}\selectfont
{\large
TEN BOLTS \\
THE DESIGN ENGINEER'S PRE-FLIGHT CHECKLIST \\
THADDEUS HUGHES \\
REV C, GENERAL USE. UPDATED: 6/5/2019 \\
INSPIRED BY TOM SACH'S `TEN BULLETS' }

\begin{outline}
\1 Utility
	\2 The component accomplishes high-level goals either directly or indirectly. Addition/revision of the component can be thoroughly justified.
	\2 All use cases of the component have been carefully considered. Additional features of a component are not simply thrown on.
	\2 A component is cool because it does this, and not for any other reason.
\1 Elegance
	\2 Efforts were made to streamline and simplify the design.
	\2 The design and execution is aesthetically pleasing on its own and flows with the rest of the system.
	\2 The beauty of the component is derived from its utility.
\1 Weight
	\2 Efforts to reduce component mass have been undertook to balance survivability and performance.
\1 Strength and Rigidity
	\2 Load paths have been thought out thoroughly and designed, not just accepted at face value.
	\2 If applicable to performance, stiffness of the component has been analyzed.
	\2 If applicable to meshing or alignment, deformation of the component has been analyzed.
	\2 Analysis has been conducted that matches physical reality as closely as possible (choosing correct boundary conditions and loads) or uses empirically derived formulations.
	\2 Fatigue has been analyzed if the component undergoes cyclical loading.
	\2 Buckling has been analyzed if there are long portions in compression.
	\2 Yield has been considered for catastrophic load cases.
\1 Survivability
	\2 Entry of rocks and conductive objects has been taken into consideration.
	\2 Entry of dirt and dust has been taken into consideration.
	\2 Entry of water and oil, both low and high temperature, has been taken into consideration.
	\2 Inertial loads have been taken into consideration.
	\2 If all bolted joints become like ball-and-socket joints, the structure is properly constrained.
	\2 The effects of component wear (eg. egging and abrasion) have been considered.
	\2 Effects of vibration have been considered and locking fasteners appropriately selected.
	\2 Overheating has been considered.
\1 Economy
	\2 All commercial over the shelf (COTS) have been considered.
	\2 Modifying COTS options or using bespoke options has been considered.
	\2 Part count is minimized. Every single fastener, adhesive, or weld is justified.
	\2 Parts are sized to readily available components rather than bespoke ones. Bespoke components are avoided.
\1 Manufacturability
	\2 In manually machined parts, complicated contours are removed, and radii are sized for available tooling.
	\2 The limitations of fabrication processes are considered.    
	\2 A ‘mental machining’ process has been gone through.
	\2 The part can be held rigidly and securely in every set-up, with minimal machine calibration.
	\2 The part can be indicated reliably in every set-up.
	\2 Tooling can reach everywhere it needs to, while being held rigidly and securely.
	\2 Effects of warping during welding are considered and mitigated.
	\2 The effects of fabrication on material properties (strain hardening, heat treatment) are taken into account.
	\2 Reasonable materials and post treatment are considered and justified.
\1 Integration
	\2 All of the following are considered and communicated:
		\3 Tolerance stackup
		\3 Loads the component receives and creates.
		\3 Vibrations the component receives and creates, along with generated noise and modes.
		\3 Thermal conditions around the component, and created by the component.
		\3 Required tolerances between components to line up and function properly.
		\3 Airflow around the component and the resulting wake.
		\3 Driver/user comfort and feedback
\1 Serviceability
	\2 Maintenance required for the component is easy to perform, practiced, and well-documented. 
	\2 The time/cost of removing/reinstalling the component has been considered. Appropriate fasteners have been selected.
	\2 Ample clearance and access has been allotted for inserting fasteners, wrenches, sockets, and hands. Conveniences like ‘wrench slots’ are integrated.
\1 Rules Compliance
	\2 All rules possibly pertaining to this component and the systems it interacts with have been searched for and complied with.
	\2 Discrepancies in rules are sorted out with the governing body promptly.
\end{outline}
\end{document}